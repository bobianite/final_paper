\documentclass[12pt, oneside, letterpaper, fleqn]{article}

\usepackage{bobianite}
\setstretch{2}

\pagestyle{fancy}
\fancyhead[L]{Bryance Oyang\\HPS 160}

\begin{document}
\begin{center}
\textbf{Transformative discoveries in physics at the turn of the 20th
century}
\end{center}

At the turn of the 20th century, physics had progressed to the point
where it was now possible to investigate the structure of matter and
light. Electromagnetic theory had been developed so that physicists
could investigate the charged particles making up the atom. Statistical
mechanics, originally developed to explain heat and gases, would help
Planck and Einstein realize the corpuscular nature of light. Physics was
ready for a major transformation. In the past, physics had been
primarily concerned with the behavior and motion of things, such as the
motion of planets, the behavior of light, the behavior of electric and
magnetic fields, or the behavior of heat and gases. In this paper, I
argue that at the turn of the 20th century, physics shifted from being
mainly concerned with the behavior of matter and light to being
interested in the actual structure of matter and light. Physicists
became interested in what made up the the things they were studying,
what was inside the atom, what was light made of, and physicists finally
had the tools they needed to find answers to these questions. I will
use the investigations into radioactivity, Thomson's discovery of the
electron, Zeeman's experiments, and Planck and Einstein's ideas about
light to illustrate the heightened interest in the structure of matter
of light during this time period.

Wilhelm R\"ontgen, in 1895, discovered X-rays, a new kind of ray that
penetrates many substances including human flesh. Inspired by
R\"ontgen's discovery, Henri Becquerel and Marie and Pierre Curie began
investigations in radioactive substances and the various rays those
substances produced. They found that radioactive substances would
spontaneously emit energetic rays and would not require an outside
source of energy to do this. These discoveries made physicists wonder
what was it about the structure of matter that allowed these rays to be
created, and also created interest in the composition of the emitted
rays.

Further investigations revealed that there were three main types of rays
emitted by radioactive substances, now called $\alpha$, $\beta$, and
$\gamma$ rays. The $\alpha$ and $\beta$ rays were able to be deflected
by magnetic fields, and physicists therefore deduced they were charged.
On the other hand, $\gamma$ rays were not deflected by magnetic fields
and behaved similarly with X-rays. Physicists hypothesized that the
emission of these rays should be associated with a ``slow modification
of the atoms of the radioactive substances'' \cite{becquerel}, but were
unsure whether the modification was brought about by additional
invisible rays or was simply spontaneous. These experiments in
radioactivity and the hypotheses developed around them demonstrates that
at the turn of the 20th century, a great interest in the structure of
matter and atomic theory was emerging.

In 1897, J.\ J.\ Thomson conducted a series of experiments to study
cathode ray tubes. There was a disagreement among English and German
physicists at the time whether these cathode rays were material
(supported by English physicists) or ethereal (supported by German
physicists) in nature. Thomson showed that the cathode rays were
deflected by both electric and magnetic fields and that they were
negatively charged. He also determined that the rays were composed of
material particles and found the charge-to-mass ratio of these
particles, now called electrons. Thomson noticed that the material of
the cathode in the tube had no effect on the charge-to-mass ratio of the
particles in the rays, suggesting that all cathode rays were composed of
the same kind of particle. 

Thomson's discovery of what made up cathode rays was considered very
important at the time, and he was awarded the Nobel Prize in Physics in
1906 for his efforts. The discovery of the electron also helped to fuel
futher interest in the structure of matter. Thomson later proposed the
plum pudding model of the atom which had electrons embedded in a sea of
positive charge.

Around the same time that Thomson was investigating cathode rays, Pieter
Zeeman investigated the effects that a magnetic field would have on the
spectral lines of atoms. Michael Faraday had previously investigated a
similar problem, but at the time he lacked the technology and equipment
to make the precise measurements that were needed, and as a result did
not find anything interesting.  Zeeman, on the other hand, was able to
discover what is now called the Zeeman effect. He vaporized some sodium
in a small container and shone a light through it. The spectrum of the
light had the dark absorption lines of sodium. Then Zeeman applied a
magnetic field to the sodium container, and he observed that the
absorption lines instantly widened.  After corresponding with Lorentz,
Zeeman published his results.  In Zeeman's publication, he does not
simply describe his observations of the effects the magnetic field had
on the spectral lines though. He puts a heavy emphasis on what his
experimental results implied about the structure of matter and boldly
hypothesizes a model of the atom, with help from Lorentz.

In this model, charged particles orbit a central nucleus in circles.
When a magnetic field is applied, the orbits are perturbed, so that the
frequency of light emitted or absorbed by the atom changes slightly. The
result is that one spectral line splits into three, and therefore looks
as if the line widened. Using Maxwell's electromagnetic theory and
assuming this model of the atom, Lorentz predicted the polarization of
light for each of the three split lines, and his prediction was found to
be in exact agreement with Zeeman's experiment. This was done around the
time, if not before the discovery of the electron by J.\ J.\ Thomson,
showing that there was an intense interest and rush to understand the
structure of matter. Furthermore, this model of the atom along with the
experiment allowed Zeeman to calculate the charge-to-mass ratio of the
electron, which was in agreement with the value obtained more directly
by J.\ J.\ Thomson.

Besides an intense interest in the structure of matter at the turn of
the century, Max Planck's explanation of blackbody radiation and
Einstein's subsequent explanation of the photoelectric effect created an
interest in the structure of light. After Maxwell formulated his Maxwell
equations for electromagnetic field, it was realized that light is
electromagnetic radiation. However, when physicists applied Maxwell's
theory and statistical mechanics to try to understand the spectrum of
blackbody radiation, they found a large disagreement between theory and
experiment. Max Planck proposed a fix to the situation, where he
hypothesized that a blackbody could only emit discrete packets of energy
as light, and the size of the packets depended on the frequency of light
being emitted. A few years later, Einstein explained the photoelectric
effect by hypothesizing that light was composed of particles called
photons, and the energy of these photons was connected to the frequency
of light in the same way as in Planck's theory.

These experiments and investigations during the turn of the century
propelled physics into a new era. The intense interest in the atom and
the nature of light created during this time fueled the quantum
revolution taking place twenty years later. Throughout the 20th century,
a large part of physics continued to investigate the structure of matter
and light. Many new particles and families of particles were discovered,
such as neutrinos, quarks, force carriers, and very recently the Higgs
boson. An important breakthrough during the turn of the 20th century
that I did not mention earlier in this paper is Einstein's theory of
relativity. It was one of the few accomplishments during this time that
was not concerned with the structure of matter or light, but was
concerned with the nature of space and time. This theory later lead to
Einstein's theory of gravitation. Interestingly, while all aspects
of physics that contributed to understanding the structure of matter and
light at the time have now been reformulated as quantum theories, such
as electromagnetic theory or statistical mechanics, Einstein's theory of
gravity, which did not come from investigating the structure of matter
or light, to this day has no satisfactory quantum counterpart.

\nocite{*}
%\bibliographystyle{unsrt}
\bibliographystyle{plain}
\bibliography{bib_paper1}

\end{document}
