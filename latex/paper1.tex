\documentclass[12pt, oneside, letterpaper, fleqn]{article}

\usepackage{bobianite}
\setstretch{2}

\pagestyle{fancy}
\headheight 1.6cm
\fancyhead[L]{Bryance Oyang\\HPS 160}

\begin{document}
\begin{center}
\textbf{title}
\end{center}

Around 1896--1897, Pieter Zeeman investigated the effects that a
magnetic field would have on the spectral lines of atoms. Michael
Faraday had previously investigated this same problem, but at the time
he lacked the technology and equipment to make the precise measurements
that were needed, and as a result did not find anything interesting.
Zeeman, on the other hand, was able to discover what is now called the Zeeman
effect. He vaporized some sodium in a small container and shone a light
through it. The spectrum of the light had the dark absorption lines of
sodium. Then Zeeman applied a magnetic field to the sodium container,
and he observed that the absorption lines instantly widened. After
corresponding with Lorentz, Zeeman published his results.
In Zeeman's publication, he does not simply describe his
observations of the effects the magnetic field had on the spectral lines
though. He puts a heavy emphasis on what his
experimental results implied about the structure of matter and boldly hypothesizes
a model of the atom, with help from Lorentz.

In this model, charged particles orbit a central nucleus in circles.
When a magnetic field is applied, the orbits are perturbed, so that the
frequency of light emitted or absorbed by the atom changes slightly. The
result is that one spectral line splits into three, and therefore looks
as if the line widened. Using Maxwell's electromagnetic theory and
assuming this model of the atom, Lorentz
predicted the polarization of light for each of the three split lines,
and his prediction was found to be in exact agreement with Zeeman's
experiment. All this was done even before the discovery of the electron by JJ
Thompson, showing that there was an intense interest and rush to
understand the structure of matter.

\nocite{*}
\bibliographystyle{unsrt}
\bibliography{bib_paper1}

\end{document}
