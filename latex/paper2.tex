\documentclass[12pt, oneside, letterpaper, fleqn]{article}

\usepackage{bobianite}
\setstretch{2}

\pagestyle{fancy}
\fancyhead[L]{Bryance Oyang\\HPS 160}

\begin{document}
\begin{center}
\textbf{P. A. M. Dirac and his Search for Mathematical Beauty in Nature}
\end{center}

At the turn of the 20th century, many physicists shifted their focus to
investigating the structure of matter and light, being inspired by the
discovery of radioactivity and the electron. Around twenty years later,
their investigations had led them to a new quantum mechanics to describe
the behavior of the atom.  Paul Dirac was one of the most creative
geniuses in theoretical physics during the quantum revolution. Dirac's
style and approach to his work has had an enormous influence on
theoretical physics. In this essay, we explore Dirac's style of physics
by following him through his many accomplishments, but to fully
understand Dirac's style of thinking, we must begin by looking at
Dirac's education and the events that influenced him.

Dirac began studying engineering at the University of Bristol when he
was 16. During his time at the university, he read in a newspaper about
the verification of Einstein's general theory of relativity, which had a
huge influence on him and sparked his interest in physics. Eager to find
out more about Einstein's theory, Dirac was disappointed when he
realized that none of his teachers knew much about relativity either.
Fortunately for Dirac, the philosopher Charlie Broad, who had studied
the Einstein's theory, was appointed as the Professor of Philosphy at
Bristol in Dirac's final year, and gave lectures on relativity. From
these lectures and from reading Arthur Eddington's book on relativity,
Dirac was able to learn both the special and general theory of
relativity.

While Einstein's theories sparked Dirac's interest in physics, Dirac's
education in engineering helped him develop an appreciation for
approximate theories, because engineers had to constantly make
approximations. Dirac realized that Newtonian mechanics can be viewed as
an approximation to Einstein's more refined relativistic mechanics, and
perhaps Einstein's theory is also an approximation to reality. Dirac
developed the belief that the purpose of science is to develop ever more
accurate approximations, and that an exact theory might never be found.
Dirac thought that ``even approximate theories can have mathematical
beauty'' \cite[pg. 45]{strangest_man} as a result of his engineering
training, and this point of view of science as approximations has
influenced many other physicists today.

Despite graduating with first class honors, Dirac was unable to find a
job after he completed his engineering degree. Recognizing Dirac's
mathematical talent, the faculty at the University of Bristol convinced
him to return to school to study mathematics. While studying mathematics
Dirac learned of the work of William Hamilton, who had reformulated
Newton's laws using a new approach, called Hamiltonian mechanics. A few
years later, Dirac would use the ideas of Hamiltonian mechanics to set up a
logical framework for quantum mechanics.

After completing his mathematics degree, again with first class honors,
Dirac got a place at Cambridge to continue his studies. Dirac's passion
was still in the theory of relativity, but Ralph Fowler was assigned to
be his supervisor. Fowler was a physicist who was an expert in atomic
theory but, disappointingly for Dirac, was not an expert in relativity.
Fowler introduced Dirac to Bohr's model of the atom and the emerging
ideas of quantum theory at the time, but Dirac continued to pursue his
interest in relativity. He attended the lectures of Arthur Eddington,
where he ``admired Eddington's mathematical approach to science, which
would become one of the most powerful influences on him'' \cite[pg.
61]{strangest_man}. This style was in contrast to that of Ernest
Rutherford, another leading physicist at Cambridge, who disliked
theorising and is noted for saying ``Don't let me catch anyone talking
about the universe in my department'' \cite[pg. 61]{strangest_man}.

Dirac was a hard-working student at Cambridge. He worked almost all day
for six days every week, reading books, journals, and reviewing his
lecture notes, only taking time off on Sundays. On Sundays, he would
take long walks that lasted the whole day, and tried not to think about
his work so that he could return to it fresh on Mondays. Dirac's
supervisor, Fowler, introduced him to atomic theory and Dirac spent many
of his days reading Arnold Sommerfeld's book on atomic theory.
Sommerfeld improved on the Bohr model of the atom by using Einstein's
relativistic mechanics instead of Newtonian mechanics to describe the
orbit of electrons around the nucleus, and Sommerfeld was able to obtain
more accurate calculations of the energy levels of the atom compared to
Bohr's original model. Dirac, being interested in relativity, must have
been influenced greatly by this work, since years later he would
merge special relativity and quantum mechanics.

In September 1925, Fowler received a paper from Heisenberg, and Fowler
promptly forwarded it to Dirac. Heisenberg presented a new view of the
atom. Heisenberg felt that the Bohr model, which talked about the
position of the electron and the time it takes for it to orbit the
nucleus, must be wrong, because these quantities could not actually be
measured in experiment. Heisenberg felt that the theory should only
establish relations between quantities that could be measured, and
developed an abstract but working theory using matrices. Heisenberg's
philosophy of only talking about things that could be measured had an
influence on Dirac, and Dirac would later write in his famous book
\textit{The Principles of Quantum Mechanics} \cite{principles} that
questions about things that cannot be investigated by experiment should
be considered as outside the domain of science, a view that has
influenced Feynman as can be seen from Feynman's lectures.

Heisenberg's theory was strange to physicists because the order in which
matrices are multiplied matters, $a$ times $b$ could be different from
$b$ times $a$, so that $ab - ba$ is not zero..  This $ab - ba$ caught
Dirac's attention, as Dirac was familiar with this noncommuting
multiplication from his mathematics training, but did not understand its
physical significance or implications. Always interested in relativity,
Dirac also tried to create a relativistic version of Heisenberg's theory
but had no success and soon gave up.

On one of Dirac's relaxing Sunday walks, it occured to him that the
noncommuting matrices in Heisenberg's theory might be related to the
Poisson bracket, a concept he encountered while studying Hamilton's
theory in classical mechanics as a mathematics student a few years ago.
However, Dirac did not remember exactly what a Poisson bracket was at
the time, and all the libraries were closed on Sunday. After waiting
impatiently and eagerly through the night, Dirac went to the libraries
on Monday morning to look up the Poisson bracket. After a few more weeks
of work, Dirac was able to complete the connection of the $ab - ba$ in
Heisenberg's work with the Poisson bracket in classical mechanics, and
realized he could build a consistent quantum mechanics through this
correspondence with classical mechanics. Dirac's supervisor Fowler
realized the importance of Dirac's work and urged the Royal Society to
quickly publish Dirac's paper. Dirac arrived at these ideas through
mathematical reasoning and analogy, as opposed to physical intuition,
showing the influence of Dirac's mathematical training. A few years
later, Dirac would return to this idea to develop quantum
electrodynamics by making the noncommutivity of quantum fields
correspond to classical Poisson brackets.

Despite using mathematics as a primary tool for his thinking, Dirac did
mathematics quite differently from mathematicians. He says ``I think you
can see here the effects of an engineering training. I just wanted to
get results quickly, results which I felt one could have some confidence
in, even though they did not follow from strict logic, and I was using
the mathematics of engineers, rather than the rigorous mathematics which
had been taught to me by Fraser'' \cite[pg. 95]{strangest_man}. Dirac's
style and approach to mathematical physics continues even today in
physics, as most physicists avoid the level of rigor that mathematicians
typically employ.

When Dirac first heard of Schrodinger's wave mechanics, he did not take
it very seriously, as he already had a working quantum mechanics from
Heisenberg and his own work. However, Heisenberg told Dirac to pay more
attention to Schrodinger's version of quantum mechanics. From reading
Schrodinger and other people's work, he realized that he could have
gotten Schrodinger's equation from Heisenberg's theory and his own
developments with Poisson brackets.

Working with Schrodinger's wavefunctions, Dirac thought about how to
represent two or more identical particles using wavefunctions. Dirac
noticed that if two identical particles exchanged places, no experiment
could possibly be done to tell that they have swapped. Following
Heisenberg's influence of only talking about things that could be
measured, Dirac realized that quantum theory should not predict a
measurable change when two identical particles are swapped. Dirac worked
out that the wavefunction must either stay the same or be multiplied by
$-1$ when two identical particles swap in order for quantum theory to be
consistent with this fact. Dirac found that fermions' wavefunctions
would be multiplied by $-1$ while bosons' wavefunctions would stay the
same. Fermions, an example being the electron, are particles that have
the property that no two fermions can occupy the same state (the Pauli
exclusion principle), and in some sense have a tendency to avoid each
other.  Bosons, an example being the photon, are particles that can have
infinitely many of them occupying the same state, and in some sense tend
to clump.

Continuing to play with the mathematics of Schrodinger's equation, Dirac
wondered what would happen if he made the wavefunction itself one of
Heisenberg's noncommuting quantities. From this line of thought, Dirac
worked out a novel mathematical way to represent fermions and bosons,
and this method is now called second quantization. Using this method, he
was able to explain Planck's blackbody radiation formula using the new
quantum mechanics by identifying bosons with oscillators. Second
quantization has become one of Dirac's greatest contributions to
physics, and it came from one of Dirac's unusual approaches to physics
of playing with equations, looking for beautiful mathematics, and seeing
what they predicted, as opposed to having a clear goal in mind and
trying to develop a method for it.

Dirac's successes in working with quantum mechanics must have given him
confidence to return to his old passion, relativity. Dirac now tackled
the problem of merging special relativity with quantum mechanics. Dirac
again resorted to playing with mathematical quantities but this time had
a clear goal which helped him limit the possibilities. He specified the
characteristics the equation must have, and used those to guide him. The
equation must agree with Einstein's special relativity, and also must
agree with his version of quantum mechanics. Dirac also believed that
the equation would be simple and beautiful. Through trial and error and
months of hard work, Dirac eventually guessed the correct equation, now
called the Dirac equation. Unlike Schrodinger's equation which only
includes one wavefunction, Dirac's equation featured four interconnected
wavefunctions. Additionally, Dirac's equation explained the existence of
spin and magnetic field of the electron as a result of the merging of
special relativity and quantum mechanics, a great triumph for
theoretical physics.

All was not well, however. Dirac's equation also predicted the existence
of negative energy electron states, something that seemed absurd and was
never observed in experiment before. This troubled Dirac greatly. He
eventually postulated that all the negative energy states were filled
somehow with undetected negative energy electrons. A vacancy in these
negative energy states would appear as a positively charged hole, and
only this hole could be detected in experiment. In 1932 at Caltech, Carl
Anderson photographed the particle track of Dirac's hole, now called the
positron. In a way, Dirac for the first time had predicted the existence
of a new particle of nature from guessing an equation that would conform
with previously known laws (quantum mechanics and relativity). This
style of theoretical physics, of guessing the mathematics based on
constraints, continued to help physicists predict the existence of more
particles, with the Higgs boson being the most recently discovered one,
but having been predicted by theory many years earlier.

Dirac's successes in producing physical theories from playing with
mathematics led him to believe that the ``most powerful method of
advance...\ is to employ all the resources of pure mathematics in
attempts to perfect and generalise the mathematical formalism that forms
the existing basis of theoretical physics, and \emph{after} each success
in this direction, to try to interpret the new mathematical features in
terms of physical entities'' \cite{monopoles}.

Following his own advice, Dirac generalized the mathematical form of
Schrodinger's wavefunctions and reinterpreted his new wavefunction as
describing particles in a magnetic field. Investigating the properties
of these generalized wavefunctions further, Dirac discovered that if
magnetic monopoles existed, it would explain the quantization of
electric charge. Physicists have only found free particles or ions with
an electric charge being an integer multiple of the charge of an
electron, but now Dirac has a theoretical explanation for this fact.
Unfortunately, his explanation rests on the existence of magnetic
monopoles, and to this day no experiment has found one yet.

Dirac had a large influence on the physicists Schwinger and Feynman.
Both had read Dirac's book \textit{The Principles of Quantum Mechanics}
during their studies, and were inspired to develop a consistent quantum
electrodynamics, which they accomplished through renormalization.
However, Dirac, always looking for mathematical beauty in physical laws,
felt the renormalization approach was ugly. He never accepted Schwinger
and Feynman's quantum electrodynamics as a satisfactory theory, despite
its successes at predicting experimental results. Dirac was never
successful developing a quantum electrodynamics that satisfied himself
either. In the last paragraph of his book, he writes, ``It would seem
that we have followed as far as possible the path of logical development
of the ideas of quantum mechanics as they are at present understood. The
difficulties, being of a profound character, can be removed only by some
drastic change in the foundations of the theory, probably a change as
drastic as the passage from Bohr's orbit theory to the present quantum
mechanics'' \cite[pg. 310]{principles}.

Dirac's style of thinking has left a significant mark in theoretical
physics. He showed the power of mathematical reasoning in discovering
new laws of nature, although his way of doing mathematics was heavily
influenced by his engineering background. Through guessing equations,
playing with mathematical objects, and seeking beautiful relationships
among these objects, Dirac made some of the greatest advances in physics
following the formulation of quantum mechanics by Heisenberg and
Schrodinger. His style and methods have continued to help physicists
create new laws and predict the existence of new particles, and we see
his influence on physics everywhere even today.

\nocite{*}
%\bibliographystyle{unsrt}
\bibliographystyle{plain}
\bibliography{bib_paper2}

\end{document}
