\documentclass[12pt, oneside, letterpaper, fleqn]{article}

\usepackage{bobianite}
\setstretch{2}

\pagestyle{fancy}
\fancyhead[L]{Bryance Oyang\\HPS 160}

\begin{document}
\begin{center}
\textbf{P. A. M. Dirac and His Search for Beautiful Mathematics}
\end{center}

Dirac's style and approach to his work has had an enormous influence
on theoretical physics. To fully understand Dirac's style of thinking,
we must begin by looking at Dirac's education and the events that
influenced him.

Dirac began studying engineering at the University of Bristol when he
was 16. During his time at the university, he read in \textit{The Times}
about the verification of Einstein's general theory of relativity, which
had a huge influence on him and sparked his interest in physics. Eager
to find out more about Einstein's theory, Dirac was disappointed when he
realized that none of his teachers knew much about relativity either.
Fortunately for Dirac, the philosopher Charlie Broad, who had studied the
Einstein's theory, was appointed as the Professor of Philosphy at
Bristol in Dirac's final year, and gave lectures on relativity. From
these lectures and from reading Arthur Eddington's book on relativity,
Dirac was able to learn both the special and general theory of
relativity.

While Einstein's theories sparked Dirac's interest in physics, Dirac's
education in engineering helped him develop an appreciation for
approximate theories, because engineers had to constantly make
approximations. Dirac realized that Newtonian mechanics can be viewed as
an approximation to Einstein's more refined relativistic mechanics, and
perhaps Einstein's theory is also an approximation to reality. Dirac
developed the belief that the purpose of science is to develop ever more
accurate approximations, and that an exact theory might never be found.
Dirac thought that ``even approximate theories can have mathematical
beauty'' \cite[pg. 45]{strangest_man} as a result of his engineering training,
and this point of view of science as approximations has influenced
Richard Feynman and many other physicists today.

Despite graduating with first class honors, Dirac was unable to find a
job after he completed his engineering degree. Recognizing Dirac's
mathematical talent, the faculty at the University of Bristol convinced
him to return to school to study mathematics. While studying mathematics
Dirac learned of the work of William Hamilton, who had reformulated
Newton's laws using a new approach, called Hamiltonian mechanics. A few
years later, Dirac would use the ideas of Hamiltonian mechanics to set up a
logical framework for quantum mechanics.

After completing his mathematics degree, again with first class honors,
Dirac got a place at Cambridge to continue his studies. Dirac's passion
was still in the theory of relativity, but Ralph Fowler was assigned to
be his supervisor. Fowler was a physicist who was an expert in atomic
theory but, disappointingly for Dirac, was not an expert in relativity.
Fowler introduced Dirac to Bohr's model of the atom and the emerging
ideas of quantum theory at the time, but Dirac continued to pursue his
interest in relativity. He attended the lectures of Arthur Eddington,
where he ``admired Eddington's mathematical approach to science, which
would become one of the most powerful influences on him'' \cite[pg.
61]{strangest_man}. This style was in contrast to that of Ernest
Rutherford, another leading physicist at Cambridge, who disliked
theorising and is noted for saying ``Don't let me catch anyone talking
about the universe in my department'' \cite[pg. 61]{strangest_man}.

Dirac was a hard-working student at Cambridge. He worked almost all day
for six days every week, reading books, journals, and reviewing his
lecture notes, only taking time off on Sundays. On Sundays, he would
take long walks that lasted the whole day, and tried not to think about
his work so that he could return to it fresh on Mondays. Dirac's
supervisor, Fowler, introduced him to atomic theory and Dirac spent many
of his days reading Arnold Sommerfeld's book on atomic theory.
Sommerfeld improved on the Bohr model of the atom by using Einstein's
relativistic mechanics instead of Newtonian mechanics to describe the
orbit of electrons around the nucleus, and Sommerfeld was able to obtain
more accurate calculations of the energy levels of the atom compared to
Bohr's original model. Dirac, being interested in relativity, must have
been influenced greatly by this work, since years later he would
merge relativity and quantum mechanics.

In September 1925, Fowler received a paper from Heisenberg, and Fowler
promptly forwarded it to Dirac. Heisenberg presented a new view of the
atom. Heisenberg felt that the Bohr model, which talked about the
position of the electron and the time it takes for it to orbit the
nucleus, must be wrong, because these quantities could not actually be
measured in experiment. Heisenberg felt that the theory should only
establish relations between quantities that could be measured, and
developed an abstract but working theory using matrices. Heisenberg's
philosophy of only talking about things that could be measured had an
influence on Dirac, and Dirac would later write in his famous book
\textit{The Principles of Quantum Mechanics} \cite{principles} that
questions about things that cannot be investigated by experiment should
be considered as outside the domain of science, a view that has
influenced Feynman as can be seen from Feynman's lectures.

What was strange to Heisenberg was that the order in which matrices are
multiplied matters, $a$ times $b$ could be different from $b$ times $a$.
This $ab - ba$ caught Dirac's attention, as Dirac was familiar with this
noncommuting multiplication from his mathematics training, but did not
understand its physical significance or implications. Always interested
in relativity, Dirac also tried to create a relativistic version of
Heisenberg's theory but had no success and soon gave up.

On one of Dirac's relaxing Sunday walks, it occured to him that the
noncommuting matrices in Heisenberg's theory might be related to the
Poisson bracket, a concept he encountered while studying Hamilton's
theory in classical mechanics as a mathematics student a few years ago.
However, Dirac did not remember exactly what a Poisson bracket was at
the time, and all the libraries were closed on Sunday. After waiting
impatiently and eagerly through the night, Dirac went to the libraries
on Monday morning to look up the Poisson bracket. After a few more weeks
of work, Dirac was able to complete the connection of the $ab - ba$ in
Heisenberg's work with the Poisson bracket in classical mechanics, and
realized he could build a consistent quantum mechanics through this
correspondence with classical mechanics. Dirac's supervisor Fowler
realized the importance of Dirac's work and urged the Royal Society to
quickly publish Dirac's paper. Dirac arrived at these ideas through
mathematical reasoning and analogy, as opposed to physical intuition,
showing the influence of Dirac's mathematical training. A few years
later, Dirac would return to this idea to develop quantum
electrodynamics by making the noncommutivity of quantum fields
correspond to classical Poisson brackets.

Despite using mathematics as a primary tool for his thinking, Dirac did
mathematics quite differently from mathematicians. He says ``I think you
can see here the effects of an engineering training. I just wanted to
get results quickly, results which I felt one could have some confidence
in, even though they did not follow from strict logic, and I was using
the mathematics of engineers, rather than the rigorous mathematics which
had been taught to me by Fraser'' \cite[pg. 95]{strangest_man}. Dirac's
style and approach to mathematical physics continues even today in
physics, as most physicists avoid the level of rigor that mathematicians
typically employ.

\nocite{*}
%\bibliographystyle{unsrt}
\bibliographystyle{plain}
\bibliography{bib_paper2}

\end{document}
